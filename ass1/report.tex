\documentclass[conference]{IEEEtran}
\IEEEoverridecommandlockouts
\usepackage{cite}
\usepackage{amsmath,amssymb,amsfonts}
\usepackage{algorithmic}
\usepackage{graphicx}
\usepackage{textcomp}
\usepackage{xcolor}
\usepackage{enumitem}
\usepackage{graphicx} %package to manage images
\graphicspath{ {.} }
\newlist{steps}{enumerate}{1}
\setlist[steps, 1]{label = Step \arabic*:}
\def\BibTeX{{\rm B\kern-.05em{\sc i\kern-.025em b}\kern-.08em
    T\kern-.1667em\lower.7ex\hbox{E}\kern-.125emX}}
\begin{document}

\title{Sequences of given length where every element is more than or equal to twice of previous\\
}

\author{\IEEEauthorblockN{1\textsuperscript{st} Sumit Kumar Sahu (IIT2019069)}
\IEEEauthorblockA{\textit{B.Tech Information Technology} \\
\textit{IIIT Allahabad}\\
}
\and
\IEEEauthorblockN{2\textsuperscript{nd} Given Name Surname}
\IEEEauthorblockA{\textit{dept. name of organization (of Aff.)} \\
\textit{name of organization (of Aff.)}\\
City, Country \\
email address or ORCID}
\and
\IEEEauthorblockN{3\textsuperscript{rd} Given Name Surname}
\IEEEauthorblockA{\textit{dept. name of organization (of Aff.)} \\
\textit{name of organization (of Aff.)}\\
City, Country \\
email address or ORCID}
}

\maketitle

\begin{abstract}
This document is a model and instructions for \LaTeX.
Please observe the conference page limits. 
\end{abstract}

\begin{IEEEkeywords}
component, formatting, style, styling, insert
\end{IEEEkeywords}

\section{Introduction}
This document is a model and instructions for \LaTeX.
Please observe the conference page limits. 

\section{ALGORITHM 1}

\subsection{ALGORITHM DESIGN}

This document is a model and instructions for \LaTeX.
Please observe the conference page limits. 

\subsection{ALGORITHM }

\textbf{Input:} N(length of sequence) and M(upper bound on sequence elements)

\textbf{Output:} Number of possible sequences of length n such that each of the next element is greater than or equal to twice of the previous element but less than or equal to m.

\textbf{Method:}

\begin{enumerate}
    \item call the solve function with parameters as n and m: solve(n,m);
    \item if $m<=0$ return 0
    \item if $n==1$ return m
    \item Declare and Initialise ret with 0.
    \item Loop the current element(i) of the sequence from 1 to m.
    \setlength{\itemindent}{+.5in}
    \item call the solve function with parameters n-1 and i/2. Add the returned value to ret
    \setlength{\itemindent}{0in}
	\item return ret;
	
\end{enumerate}


\subsection{APRIORI ANALYSIS }

This document is a model and instructions for \LaTeX.
Please observe the conference page limits. 

\subsection{APOSTERIORI ANALYSIS }

First we Fix M=300 and vary N from 1 to 7
\begin{center}
   \begin{tabular}{|c|c|}
   \hline
   N & Time \\
   \hline\hline
    1 & 0 \\
    \hline
    2 & 3 \\
    \hline
    3 & 224 \\
    \hline
    4 & 4493 \\
    \hline
    5 & 34892 \\
    \hline
    6 & 112558 \\
    \hline
    7 & 246904 \\
    \hline
    \end{tabular} 
\end{center}

\includegraphics[width=10cm, height=10cm]{Figure_1.png}

Now we Fix n=4 and vary M from 1 to 100

\begin{center}
   \begin{tabular}{|c|c|}
   \hline
   N & Time \\
   \hline\hline
    1 & 0 \\
    \hline
    16 & 0 \\
    \hline
    30 & 3 \\
    \hline
    40 & 7 \\
    \hline
    50 & 13 \\
    \hline
    60 & 22 \\
    \hline
    100 & 99 \\
    \hline
    \end{tabular} 
\end{center}

\includegraphics[width=8cm, height=8cm]{Figure_2.png}

\section{ALGORITHM 2}

\subsection{ALGORITHM DESIGN}

This document is a model and instructions for \LaTeX.
Please observe the conference page limits. 

\subsection{ALGORITHM }

\textbf{Input:} N(length of sequence) and M(upper bound on sequence elements)

\textbf{Output:} Number of possible sequences of length n such that each of the next element is greater than or equal to twice of the previous element but less than or equal to m.

\textbf{Method of isZero(n,m)}

\begin{enumerate}
    \item if $n>=60$ return false;
    \item if $m<2^{n-1}$ return false;
\end{enumerate}

\textbf{Method of solve(n,m)}

\begin{enumerate}
    \item Check if the answer is 0. If found 0 return.
    \item if n==1 return m
    \item if mem[n][m]!=-1 return mem[n][m]
    \item Call solve for n,m-1 and n-1,m/2 and assign their sum in mem[n][m]
	
\end{enumerate}

\subsection{APRIORI ANALYSIS }

This document is a model and instructions for \LaTeX.
Please observe the conference page limits. 

\subsection{APOSTERIORI ANALYSIS }

First we Fix N=10 and vary M from 1 to 100000

\begin{center}
   \begin{tabular}{|c|c|}
   \hline
   M & Time \\
   \hline\hline
    1 & 0 \\
    \hline
    2001 & 180 \\
    \hline
    9001 & 224 \\
    \hline
    43001 & 4282 \\
    \hline
    78001 & 7857 \\
    \hline
    88001 & 8701 \\
    \hline
    99001 & 9704 \\
    \hline
    \end{tabular} 
\end{center}

\includegraphics[width=10cm, height=10cm]{Figure_3.png}

Now we Fix m=1000000 and vary n from 1 to 100

\begin{center}
   \begin{tabular}{|c|c|}
   \hline
   N & Time \\
   \hline\hline
    1 & 12205 \\
    \hline
    16 & 130887 \\
    \hline
    30 & 0 \\
    \hline
    40 & 0 \\
    \hline
    50 & 0 \\
    \hline
    60 & 0 \\
    \hline
    100 & 0 \\
    \hline
    \end{tabular} 
\end{center}

\includegraphics[width=8cm, height=8cm]{Figure_4.png}

\section*{CONCLUSION}

The preferred spelling of the word ``acknowledgment'' in America is without 
an ``e'' after the ``g''. Avoid the stilted expression ``one of us (R. B. 
G.) thanks $\ldots$''. Instead, try ``R. B. G. thanks$\ldots$''. Put sponsor 
acknowledgments in the unnumbered footnote on the first page.

\begin{thebibliography}{00}
\bibitem{b1} G. Eason, B. Noble, and I. N. Sneddon, ``On certain integrals of Lipschitz-Hankel type involving products of Bessel functions,'' Phil. Trans. Roy. Soc. London, vol. A247, pp. 529--551, April 1955.
\bibitem{b2} J. Clerk Maxwell, A Treatise on Electricity and Magnetism, 3rd ed., vol. 2. Oxford: Clarendon, 1892, pp.68--73.
\end{thebibliography}

\end{document}

